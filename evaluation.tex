\section{Cubedate Evaluation and Discussion}
\label{sec:evaluation}

The analysis of the implementation was two-fold, first we evaluate how the SUIT
based architecture fitted the CubeSat and CubeSat hosted payload use case.
Second, we discuss possible modifications to the workflow, architecture and
protocols.

\subsection{Evaluation}

The hardware setup for the numerical analysis consisted of:

\begin{itemize}
    \item STM32F401RE, featuring a CORTEX\-M4 core with 512Kbytes of Flash memory
          and 96Kbytes of RAM, running at 84Mhz.
    \item MCP2515 Stand-Alone CAN Controller, running at 500Kbps.
    \item 16Gbytes SD-CARD as an external storage for the CubeSat OBC simulator
\end{itemize}

In the following, code measurements where generated compiling with ARM GCC 10.2.1,
optimized for code size.

\subsubsection{Memory Footprint}

To evaluate the RAM \& ROM footprint cost of the CubeDate architecture we compared
those footprint across 4 scenarios:

\begin{itemize}
    \item \textbf{bootloader}: RIOT default bootloader application incremented to
    verify a sha256 digest of each firmware image.\footnote{Reason for the differences
    with numbers with the bootloader application in\cite{zandberg2019secure}}
    \item \textbf{baseline}: a minimal CSP application, CSP network stack and CAN-BUS
    interface.
    \item \textbf{OTA}: a minimal OTA application, CoAP endpoint, CSP network stack
    and CAN-BUS interface.
    \item \textbf{CubeDate}: CubeDate setup: CoAP endpoint, SUIT update of FW(ROM)
    and VirtualMachine\cite{zandberg2021femto}(RAM), CSP Network stack and CAN-BUS interface.
\end{itemize}

Furthermore we categorize the different components in the firmware as follows:

\francisco{need to improove on categorization some stuff was not fitted in}
\begin{itemize}
    \item can: CAN (Controller Area Network) stack as well as low level interface
    \item cbor\&cose: libraries used for COSE and CBOR parsing
    \item crypto: all cryptographic algorithms such as digest algorithm, digital
    signature, ECC and bignum, as well as pseudo-random numbers generator
    \item core: everything else??
    \item coap: CoAP protocol library (CoAP endpoint stack excluded)
    \item network: CSP (Cube Sat Protocol) network stack
    \item ota: components enabling retrieval and installation of delivered data
    (fw or other), this include e.g. the CoAP endpoint stack.
    \item suit: code related to parsing a SUIT manifest
\end{itemize}

The flash memory footprints (total and broken down per component) are shown in
\ref*{tab:footprint-rom}. The crypto library used is D.Beer C25519 library, this
is the library with the lowest footprint for ED25519 as shown in \cite{zandberg2019secure}.

\begin{table}[ht]
    \caption{ROM Footprint STM32F401RE}
    \label{tab:footprint-rom}
    \centering
    \includestandalone[width=1\columnwidth]{Figures/texfigs/flash_footprint_table}
\end{table}
\begin{table}[ht]
    \caption{RAM Footprint STM32F401RE}
    \label{tab:footprin-ram}
    \centering
    \includestandalone[width=1\columnwidth]{Figures/texfigs/ram_footprint_table}
\end{table}

\begin{table}[ht]
    \caption{SUIT Manifest Binary Overhead}
    \label{tab:manifest-overhead}
    \centering
    \includestandalone[width=1\columnwidth]{Figures/texfigs/manifest_overhead_table}
\end{table}

\subsubsection{Energy Footprint}

\francisco{Didier Olivier can you add some numbers}

\subsubsection{Network Stack}

Standarized network stack usage: \cite{wachter20206locan01}, numbers on gnrcminimal,
if Using LoRaWAN SCHC\cite{rfc8724}, is it worth using a custom network stack?

\subsubsection{SUIT Data}

Size of payloads, containerization, differential updates(vcdfiff), leverage mission files.

\subsubsection{SUIT Manifest}

Taking a look at \ref*{tab:manifest-overhead} suggest stripping down some things.

\subsubsection{SUIT Authentication}

SUIT relies on COSE for authentication, but although digital signatures is the
most used method, the standard also for using tagged Message Authentication Code
(MAC).

The FLASH cost of crypto shown in \ref{tab:footprint-rom} comes mostly from the
digital signature library 75\%\cite{zandberg2019secure}. If instead an HMAC-256/64
MAC authentication algorithm was used the ROM footprint can be greatly reduced.
This would also allow to reduce he size of the Authentication block\ref*{tab:manifest-overhead}
Granted, a shared secret would require being distributed before hand, but for a
CubeSat use case where an attacker would have no-likely physical access to the
device (once in-orbit), the reliance on the shared secret might be an adequate
compromise.
