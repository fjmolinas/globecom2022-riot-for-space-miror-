\section{Cubedate Evaluation and Discussion}
\label{sec:evaluation}

%\subsection{Evaluation}

In the following, code measurements where generated compiling with ARM GCC 10.2.1,
optimized for code size. As code base, we used RIOT release 2022.01 and SUIT configured
with ed25519 digital signatures provided by the C25519 crypto library (which has a small
footprint as shown in prior work~\cite{zandberg2019secure}).

\subsubsection{Memory Footprint Overhead}

To evaluate the RAM and Flash footprint of our Cubedate implementation, we apply it to
the ThingSat use-case, compiled for the hosted payload hardware described in \autoref{sec:thingsat-hw}.

In \autoref{tab:footprint} we compare the RAM and Flash memory requirements for a ThingSat firmware with/without Cubedate-compliant updates.
We observe that Cubedate requires a memory budget of $\sim$4KB of RAM and $\sim$19KB of Flash, which represents roughly a 10\%
increase in the total RAM and Flash memory budget for ThingSat.

\subsubsection{Network Transfer Overhead}
On the UHF/VHF link at 1kb/s, the additional network transfer time induced by the Cubedate firmware size overhead (19KB) is roughly 15 seconds.
This overhead is reasonable, but non-negligible considering that a connection to the CubeSat is segmented in time windows of $\sim$300 seconds.

Next, in \autoref{tab:footprint} we look in more details at the metadata (SUIT manifest) used to secure Cubedate software updates for ThingSat.
As we can see, the metadata including all CBOR/COSE formatting, digests (SHA256 hashes) and authentication data (ed25519 signature) amounts to $\sim$330B.
The metadata overhead thus incurs negligible overhead (+0,15\%) in case of a ThingSat firmware update (of size 200KB on average, recall
\autoref{sec:thingsat-update-req}). However, for smaller software update such as updating a mission scenario (average size 700B) the metadata
overhead is significant (almost +50\%). Nevertheless, on the UHF/VHF link at 1kb/s, this overhead remains negligible in terms of additional
network transfer time.

\iffalse

architecture we evaluate them
in two configurations, furthermore we categorize the different components to
distinguish the SUIT endured overhead

\begin{itemize}
    \item \textbf{ThingSat} refers to the ThingSat payload application with no
    software updates support
    \item \textbf{Cubedate} implementation of the Cubedate architecture on the
    ThingSat payload
    \item CAN (Controller Area Network) stack as well as low level interface
    \item Crypto includes all cryptographic algorithms such as digest algorithm, digital
    signature, ECC and bignum, as well as pseudo-random numbers generator
    \item CoAP protocol library (CoAP endpoint stack excluded)
    \item CSP (Cube Sat Protocol) network stack
    \item LoRa GW includes the sx1302 driver as well as the gateway code
    \item SUIT englobe all components enabling retrieval and installation of suit data
    (fw or other), this include e.g. the CoAP endpoint stack.
    \item Firmware: application specific code related to the CubeSat Payload excluding
    the LoRa gateway
\end{itemize}

The flash memory footprints (total and broken down per component) are shown in
\autoref{tab:footprint}. It can be seen that the overhead of Cubedate for ThingSat
is of $\sim$4KB of RAM and $\sim$19KB of flash.

\fi

\begin{table}[ht]
\begin{adjustbox}{width=0.8\columnwidth,center}
    \centering
    \includestandalone[width=1\columnwidth]{Figures/texfigs/memory_footprint_table}
\end{adjustbox}
\caption{Cubedate implementation: memory footprint in bytes.}
\label{tab:footprint}
\end{table}

\section{Discussion \& Perspectives}

\subsubsection{Portability} Bolted on top of RIOT, our Cubedate implementation works out-of-the-box
(or is trivially portable) on a very wide variety of low-power hardware built on Cortex-M microcontrollers:
the bulk of the 200+ boards supported by RIOT. However, additional work would be needed to support
other hardware based on different 32-bit microcontroller architectures (e.g. RISC-V).

\subsubsection{Network Stack Simplification \& Standardization}
The use of CSP was mandated by the CubeSat operator. The purpose of CSP was to provide an ultra-low
footprint equivalent of the IP protocol stack for CubeSats with legacy (8-bit) microcontrollers.
However, on modern (32-bit) microcontrollers such as those used in ThingSat, this approach is can be discussed.
More widely spread standard alternatives to CSP seems possible for a similar "price".
For example, RIOT's default low-power IPv6 (6LoWPAN) stack used with static routing has a footprint
in RAM/Flash memory that is comparable to libCSP memory footprint. The 6LoWPAN stack
could run directly on the CAN bus or on LoRa (see 6loCAN~\cite{wachter20206locan01}
and SCHC~\cite{rfc8724}).

\subsubsection{Alternative cryptographic primitives}
In our experimental evaluation, we used ed25519 (elliptic curve cryptography) digital signature
providing 128-bit pre-quantum security. One can consider either alternative primitives, while
remaining compliant with Cubedate and the SUIT standard. One compromise to significantly decrease
network transfer and memory footprint with Cubedate is to use HMAC (symmetric crypto) instead of
digital signatures for authentication. Another option would be to upgrade security to 128-bit
post-quantum security by using hash-based signature instead of ed25519.
