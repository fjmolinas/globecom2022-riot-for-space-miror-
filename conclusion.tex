\section{Conclusion}
\label{sec:conclusion}
As the space race intensifies, rises the need for state-of-the-art security to protect software updates on multi-tenant cubesat in orbit.
In this paper, we provided a corresponding case-study: ThingSat, a low-power payload we designed, hosted on a cubesat operated by a separate entity, currently orbiting.
We then designed and implemented Cubedate, a framework achieving strong security guarantees and low overhead, for continuous deployment of software over the air on multi-tenant cubesat.
The open source implementation of Cubedate we provided and evaluated for ThingSat was built to be reusable on a wide variety of low-power cubesat hardware.

\iffalse
This work studied ThingSat, demonstrating the increasing need for in-orbit software
updates with high security guarantees. We provide these with CubeDate, a
software-update architecture leveraging open-standards and open-source, to fulfill
the security contract and maintaining a low-entry barrier. It leverages SUIT independent
of the chosen software and hardware stack, and implemented it for a common CSP/CAN-BUS stack.
We showed that the memory and transmission overhead, is small for medium-sized
updates (firmware) and bearable for small-sized updates (mission-logic), with
further optimizations possible. However, room for improvement remains
in developing open-standards: SUIT, to provide equal guarantees while trimming-down
non-critical update-metadata for highly constrained links such as LoRa.
\fi
