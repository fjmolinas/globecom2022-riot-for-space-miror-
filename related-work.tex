\section{Background \& Related Work}
\label{sec:background-related-work}

\subsection{CubeSat}
\label{sec:cubesat}
\paragraph*{Architecture}
% - electronics
% - connectivity
% - payloads, obc, etc..
\paragraph*{Context}: largely educational, learning platform

% \subsection{Cube Sat Protocol (CSP)}
% \label{sec:csp}
\subsection{Firmware and Software Updates}
\label{sec:fu-cubesat}

\paragraph*{Challenges of firmware updates in space}: link budget, memory corruption
due to space radiation, high latency, etc.
\paragraph*{Microcontroller}: studies issues related to firmware updates of in
orbit satellite subsystems \cite{sunter2016updatesnano}.
\paragraph*{Linux systems}: most of the work in literature has focused on OBC firmware
updates, which tend to be linux systems. Most solutions target on-board storage
and error correction. \cite{FitzsimmonsReliableSoftwareUpdates}
\paragraph*{ThingSat Differences}: noy only firmware updates but also isolated business
logic or software updates of individual subsystems components. Educational missions
often a small part of the business logic needs to be updated.
\paragraph*{Authenticity of Delivered updates}: as tenant, no control over deployment
chain, risk of hacking, compliance with regulations, etc.
\paragraph*{Benefits of MCU payload}: power consumption (how much
on time as a Satellite tenant).

\subsection{Open Source} \cite{shalashov2021OpenSourceCubeSatReview}, \cite{Holliday2019PyCubed}
\paragraph*{Why Open-Source Satellite Initiatives}
\paragraph*{pyCube, etc.}
% - overview of alternatives
% - value of open source leveraging alternatives
\paragraph*{RIOT}: project overview, architecture overview\cite{baccelli2018riot}


\subsection{LoRa}
\label{sec:lora-cubesat}
\paragraph*{CubeSat LoRa missions} \cite{saeed2020CubeSatReview}
\paragraph*{Research topics, challenges of LoRa in space} \cite{saeed2020CubeSatReview}
