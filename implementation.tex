\section{Cubedate Implementation}
\label{sec:implementation}

We implemented Cubedate by extending the open source ecosystem around RIOT, which we combined with
our initial firmware running on the ThingSat payload, also open sourced, which we published and
maintain at \cite*{git:cubedate-repo}/

\underline{Communication Bus}
To provide a simple socket-like API on top of the raw CAN bus connecting hosted
payloads and OBC on the CubeSat, we used \href{https://github.com/libcsp/libcsp}{libCSP},
a library implementing the CubeSat Space Protocol (CSP version 1), which we integrated into RIOT.
This stack is deployed on Satellites such as Nuts\cite{birkeland2014nutsoverview}, GomSpace
and used by SatRevolution, as it was a requirement for communicating with the OBC.
Roughly, CSP provides an \"equivalent\" of the IP/UDP stack and static routing.

\underline{SUIT implementation :}
To fully support the wide variety of software updates case required by Cubedate, we
extended existing RIOT implementation of SUIT~\cite{zandberg2019secure} which,
initially, supported only firmware updates, internal storage, and WPAN network delivery.
We generalized the SUIT state machine to add support for:

\begin{itemize}
    \item \textbf{Heterogeneous update delivery mechanisms}: configurable \{message model,
    network stack, network interface\} bundle, or FileSystem \textit{read/write} functions.
    \item \textbf{Heterogeneous storage destination}: configurable internal/external memory:
    Volatile (e.g. RAM for mission files, or for tiny runtime execution containers such as
    FemtoContainers~\cite{zandberg2021femto})or non-volatile (e.g. FileSystem or internal Flash).
    \item \textbf{Heterogeneous update data URI}: either a local file (e.g.: mounted USB device)
    or a remotely accessible file (eg CoAP or HTTP endpoint);
\end{itemize}

\underline{Pull implementations:}
We implemented several alternatives for the \texttt{client/server} interaction which
must be initiated by the hosted payload to pull (GET) software update data over the
network from either the OBC, or a (LPWAN) groundstation.

The first pull mechanism we implemented uses CoAP~\cite{rfc7252}, an open standard
low-power IPv6 protocol following a general-purpose REST architecture. In particular,
it is not only applicable over the CAN/CSP link, but also usable over VHF links~\cite{palma2018vhf-coap}
or LoRa links~\cite{sanchez2018lora-coap}, which paves the way for some Cubedate
software updates over the direct link from a ThingSat LoRa groundstation to the ThingSat payload.

However, Cubedate is agnostic w.r.t the pull mechanism. To demonstrate this, we
also implemented an alternative pull mechanism using UP (the Universal Platform protocol),
a dedicated proprietary protocol provided by the CubeSat operator for \texttt{Request/Response}
interaction with the OBC. This option is currently deployed and running on ThingSat in orbit.
