\section{Cubedate Implementation}
\label{sec:implementation}

\iffalse
SUIT is used in Cubedate payloads as a \textit{Data Delivery Mechanism} bundling
the security properties described in section IV D. The implementation described
next seeks to remain generic to be re-usable in different scenarios while tailored
for the Satellite ThingSat use case.

SUIT state machine running on the deployed device does not need to care about
how this \textit{Data} is delivered or where it will be installed, it just
needs to be handed a \textit{manifest}.

\begin{itemize}
    \item \textbf{Data Resource URI}: a locally (e.g.: mounted USB  device) or remotely
    (HTTP or CoAP endpoint) accessible file.
    \item \textbf{Data Resource Delivery Mechanism}: transport mechanism to deliver SUIT:
    e.g. \{message model, network stack, network interface\}
    bundle, or FileSystem \textit{read/write} functions.
    \textit{manifest} and data resource to the SUIT state machine.
    \item \textbf{Data Resource Installation Storage}: internal or external Volatile or
    Non-volatile storage, e.g. RAM (mission files, FemtoContainers\cite{zandberg2021femto})
    or FileSystem or internal FLASH.
\end{itemize}

As the case where the Data Resource URI points to a locally available is simply a
simplified case, a networked scenario with a URL will be described.

\francisco{PLACE HOLDER FOR A DIAGRAM}
\fi



\subsection{Message Model, Request/Response Semantics}

A \texttt{client/sever} model adequately represents the interactions between the SUIT state
machine (ran in the \textit{payload}) and the Data Resource. In the scope of the
update process the \textit{payload} will act as the client requesting a \textit{manifest}
and sequently the \textit{Data Resource}, but in the global picture it will act
as both \texttt{client \& server}, answering to queries from the OBC or the
ground-segment. ThingSat uses UP \francisco{some info on UP}, a proprietary message
model enabling request-response type interactions, which can be replaced by a
generic \texttt{Request/Response} layer.

\texttt{CoAP} is therefore used because its designed to run on a connection-less transport
and offers little overhead in the exchanged messages as well as support for exchanging
messages larger than the underlying protocols MTU (\francisco{cite block coap}
(4 \texttt{bytes} \texttt{CoAP} header, 0-3 extra \texttt{bytes} for block transmissions).
Using a standard like \texttt{CoAP} allows for simplified integration into \textit{North-bound}
application code, e.g.: \texttt{SUIT} as well as allowing to easily replace the
\texttt{South-bound} network-stack.

Note that a \texttt{Request/Response} approach is not a requirement, but makes
for a clean architecture.

\subsection{Network Stack}

A CubeSat Network will commonly be split into two segments: an in-orbit segment
(OBC \& \textit{payloads}) and a Ground-segment (Ground Station). This motivated
the existence of the \textit{Cubesat Space Protocol CSP} deployed on Satellites
such as Nuts, GomSpace, etc. \francisco{cite}. But recent deployments see CubeSats
as part of larger Network such as LoRaWan \francisco{cite fossa}, and IP based
networks are also present on CubeSats.

\textit{CSP} usage was a requirements specified from SatRevolution as the network
stack to use for ThingSat. Nonetheless this implementation aspired to be agnostic
in design to the network stack, allowing to easily replace it by e.g. an
\texttt{IPV6/UDP} or \texttt{LoRaWAN} stack.

This was done by using a {\bf A generic North-bound interface} called \texttt{sock}
which provides a socket-like API for network stacks. Although designed for IP,
it can be wrapped around CSP if care is taken to handle addressing.

\subsection{Communication Bus}

\texttt{LibCSP} and \textit{CSP} support several physical layers such as CAN, I2C,
RS-232, network interfaces wrappers also exist for TAP, ZMQ, and can be easily
extended to cover the radio link between the \textit{Satellite} and
\textit{Ground-Segment}. The physical layers supported are not random, they match
the most common serial communication interfaces found on \textit{Satellite}
payloads, OBC and components. Ethernet is also commonly available  but prohibitive
in most microcontrollers based use-case because of its high power consumption.
\francisco{add some details on CAN?}

CAN was used since it provides an extremely robust, simple and efficient,
widely proven in the automotive, flight and space industry.

\subsection{Operating System: RIOT}

RIOT is a general-purpose OS designed for small IoT devices based on micro-controllers.
It provides with blah blah..

\subsubsection{Integrating LibCSP and RIOT}

Details on the integrations work, learned lessons and sufferings.

\subsection{Continuous Integration \- Collaborative Platform}

The CubeDate architecture was implemented on ThingSat (using UP instead of CoAP).
A Continuous Integration collaborative platform mimicking the update workflow described
in section \francisco{section?}. This consist on a minimal implementation of the
\texttt{OBC} - \texttt{payload} interactions.

The \texttt{OBC} can be ran on RIOTs \texttt{native} platform with a mounted
FileSystem or on any RIOT \texttt{board} with an `sdcard` slot or adapter.
\textit{manifests} and matching \textit{Data-Resources} can be pushed to the
\texttt{OBC} which in its turn acts as a \texttt{CoAP} file-server.

The Payload can run on RIOTs \texttt{native} platform or on any \texttt{RIOT}
\texttt{board}. Different \textit{Data-Resources} can be delivered to RAM,
External NVM or internal NVM (FLASH), simulating delivery of mission files,
firmware updates or other.

Both \texttt{OBC} and \texttt{payload} run over \texttt{LibCSP} over CAN. If
using \texttt{native} \texttt{SocketCAN} is used to create virtual can
devices.

\francisco{Place Holder of Diagram of the CI}

The developments related to this paper where in release in open-source form for
the benefit of CubeSat and RIOT communities and can be found at.

