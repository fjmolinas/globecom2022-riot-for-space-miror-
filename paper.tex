\documentclass[conference]{IEEEtran}
% Some Computer Society conferences also require the compsoc mode option,
% but others use the standard conference format.
%

\usepackage[pdftex]{graphicx}
\graphicspath{{./Figures}}

% *** CITATION PACKAGES ***
%
%\usepackage{cite}
% cite.sty was written by Donald Arseneau

\usepackage{comment}
\usepackage{mcite}
\usepackage{cancel}
\usepackage{blkarray}
\usepackage[export]{adjustbox}
\usepackage{tikz}
\usepackage[mode=buildnew]{standalone}% requires -shell-escape
\usepackage{pgfplots, pgfplotstable}
\pgfplotsset{compat=1.16}
\usepackage{lipsum}
\usepackage{colortbl}
\usepackage{csvsimple}
\usepackage{adjustbox}
\PassOptionsToPackage{hyphens}{url}\usepackage{hyperref}

\newcommand{\francisco}[1]{\textbf{\textcolor{blue}{[Francisco] #1}}}
\newcommand{\nath}[1]{\textcolor{green}{#1} }
\newcommand\todoEB[1]{\textcolor{red}{EB: #1}}
\newcommand\todoOA[1]{\textcolor{red}{OA: #1}}

% *** MATH PACKAGES ***
%
\usepackage{amsmath}

% *** SUBFIGURE PACKAGES ***
\ifCLASSOPTIONcompsoc
  \usepackage[caption=false,font=normalsize,labelfont=sf,textfont=sf]{subfig}
\else
  \usepackage[caption=false,font=footnotesize]{subfig}
\fi

\begin{document}
\bstctlcite{IEEEexample:BSTcontrol}

%\title{ThingSat, Case Study for In-Orbit Firmware and Software Updates Architecture}
\title{Cubedate: Securing Software Updates in Orbit \\ for Low-Power Payloads Hosted on CubeSats}

% author names and affiliations
% use a multiple column layout for up to three different
% affiliations
\author{
%\IEEEauthorblockN{J. Kirk, H. Simpson and M. Scott}
\IEEEauthorblockN{Didier Donsez, Olivier Alphand}
\IEEEauthorblockA{Grenoble Alpes University, France}
\and
\IEEEauthorblockN{François-Xavier Molina, Koen Zandberg, Emmanuel Baccelli}
\IEEEauthorblockA{Inria, France}
}

% make the title area
\maketitle

% As a general rule, do not put math, special symbols or citations
% in the abstract
\begin{abstract}
CubeSat design is facilitated by the increasing availability of open-source software in the domain, and a variety of low-cost
hardware blueprints based on commodity microcontrollers. We attain the rock-bottom price to reach orbit as entities that design,
launch and operate CubeSats started selling to multiple tenants tiny rack slots (typically 0,25U each) for low-power payloads
that may be hosted on their CubeSat. The question arises of how to provide state-of-the-art security for software updates on
a multi-tenant CubeSat, whereby mutual trust between tenants is limited. In this paper, we provide a case- study: ThingSat,
a low-power payload we designed, is currently hosted on a CubeSat orbiting at 500km altitude operated by a separate entity.
We then design Cubedate, a framework for securing continuous deployment of software to be updated on orbiting multi-tenant
CubeSats. We also provide a highly portable open-source implementation of Cubedate, based on the IoT operating system RIOT,
which we evaluate experimentally.
\end{abstract}

\IEEEpeerreviewmaketitle



\section{Introduction}
\label{sec:introduction}

\paragraph*{Satellites, CubeSat}
\paragraph*{IoT, Internet of Space Things}: Iridium, LoRa, 5G
\paragraph*{Cube-Sat, Open-Source Initiatives}
\paragraph*{RIOT, Open-Source}
\paragraph*{Paper contributions}
\paragraph*{Paper organization} 


\section{Background \& Related Work} 
\label{sec:background-related-work}

\subsection{CubeSat} 
\label{sec:cubesat}

\subsection{Cube Sat Protocol (CSP)} 
\label{sec:csp}

\paragraph*{Architecture}
% - electronics
% - connectivity
% - payloads, obc, etc..

\subsection{Open Source} \cite{shalashov2021OpenSourceCubeSatReview}, \cite{Holliday2019PyCubedAO}
\paragraph*{Why Open-Source Satellite Initiatives}
\paragraph*{pyCube, etc.}
% - overview of alternatives
% - hardware software overview

\subsection{LoRa} \cite{saeed2020CubeSatReview} 
\label{sec:lora-cubesat}
\paragraph*{CubeSat LoRa missions}
\paragraph*{Research topics, challenges of LoRa in space}

\subsection{RIOT} \cite{baccelli2018riot}
\paragraph{Overview of the Project}
\paragraph{Overview of the architecture}

\section{ThingSat}
\label{sec:case-study}

\subsection{Overview}
\paragraph*{Mission Goal:} in-orbit observation of glaciers in Europe and in Polynesia.
\paragraph*{Approach}: rent "rack" space as payload on a CubeSat (SatRevolution).
\paragraph*{High-level overview of Segments}

\subsection{Distributed System Architecture}
\paragraph*{Ground Segment Description}
\paragraph*{Space Segment Description}
\paragraph*{Control Segment} % Francisco: not sure if its the right name
\paragraph*{OBC \& Payload Communication Bus}
\paragraph*{Payload Description}: tenant status, energy budget, on time etc.
% - OBC -> SatRev controlled
% - Payload -> Full Control

\subsection{Communication Characteristics}
\paragraph*{link-budget Mission Control}: delivering and communicating mission
files, delivering software updates, latency, etc.
\paragraph*{link-budget Mission}: communication between OBC and Payload, payload
active time for communication with ground segments.

\subsection{Software Updates Requirements}
\paragraph*{What needs updates}: ?
\paragraph*{Why?}: timeline for deployment, CoVID context, etc. making it impossible
to deliver a fully functional device in time. Need to update the firmware remotely
and in orbit.
\paragraph*{Communication Chain}: how are new firmware, payloads delivered,
who has access, etc..


\section{Cubedate: Standard Security for Continuous Deployment of Low-power CubeSat Software}
\label{sec:low-power-orbital-communication-arch}


The minimal security guarantees that we aim for are authenticity and integrity of software udpates, during the lifetime of the satellite mission.
These guarantees must remain valid end-to-end, all the way from the hosted payload software maintainer to the payload hosted in orbit on the CubeSat.
The basic process we use for securing authenticity and integrity of software udpates is decomposed in six phases shown \autoref{fig:phases}. 

\subsection{Basic Software Life-cycle Phases}

During a preliminary, pre-flight phase (\textit{Phase 0}) the authorized maintainer for the CubeSat-hosted payload
produces and flashes the payload with commissioning material:
a bootloader, the initial firmware, and authorized crypto material (a public key, and a crypographically strong hash function).
Once the hosted payload is commissioned it can be sent to the CubeSat operator of installation in the CubeSat.


Once the CubeSat in orbit, the hosted payload maintainer can trigger iterations through cycles of Phases 1-5, whereby
the authorized maintainer can build a new software update (\textit{Phase 1}), hash the update
and sign the hash (\textit{Phase 2}) then push a network transfer (PUT) towards the hosted payload via the ground station and the OBC (\textit{Phase 3.1}). The next time it wakes up, the hosted payload can
then ping and fetch (GET) the update from the OBC (\textit{Phase 3.2}), proceed to verify the signature and the hash (\textit{Phase 4}),
and upon successful verification, install/boot the new software (\textit{Phase 5}), otherwise the update is dropped.

\begin{figure}[t]
    \centering
    \includegraphics[width=0.5\textwidth]{Figures/CubeSat-Payload-update.png}
    \caption{CubeSat hosted payload secure software update process.}
    \label{fig:phases}
\end{figure}

\subsection{Supporting Network Transport Heterogeneity}
This aspect concerns both \textit{Phase 3.1} and \textit{Phase 3.2} in \autoref{fig:phases}. 

First of all, software updates may be transported over a wide variety of network segments.
\begin{itemize}
\item {\bf Developer to Groundstation}: via wired Internet, USB key upload...
\item {\bf Groundstation to CubeSat}: via radio links such as UHF, LoRa...
\item {\bf Intra-CubeSat}: via bus communication such as CAN/CSP, SPI...
\end{itemize}

Second, paths across the network may vary in complexity.
In the simplest case the end-to-end network path covers only the segment from a ground station to the OBC (or directly to the hosted payload via its own radio interface, if it has one as described in section XXX).
In more complex cases involving hosted payloads, not only
must the end-to-end path traverse heterogeneous network segments,
but also: this path may never exist end-to-end at any point in time.
This Delay-tolerant network characteristic is due to power-off periods imposed on CubeSat hosted payloads, 
combined with orbiting CubeSat being out-of-range most of the time via radio w.r.t. available ground stations.
Hence, while in transit across the network, software update data may have to be temporarily cached at some intermediate node along the path.

To cope with this wide variety of scenarios, different solutions may be used at the network, transport and application layers. \todoEB{Add here some short text about what ThingSat does, also hint at exotic things such as ICN?}. 


However, Cubedate does not specify the use of any particular approach at the network, transport and application layers to enable the delivery of software updates across the network.
Cubedate only aims to guarantee end-to-end security properties for the software update binaries delivered over the network.

\subsection{Supporting Updated Software Heterogeneity}
This aspect concerns both \textit{Phase 1} and \textit{Phase 5} in \autoref{fig:phases}. 
Software that must be updated may be of various nature, and size.
(1 )Firmware updates, (2) Mission/configuration files, (3) things like Femto-containers?.

\subsection{Low-power Standardized End-to-End Security}

\todoEB{Describe and motivate here use of of SUIT metadata.}

Using Cubedate and the SUIT standard, software updates for payload hosted on CubeSats mitigate attacks including:

\begin{itemize}
\item {\bf Tampered Software Update Attacks –} An attacker may try to update the IoT device with a modified and intentionally flawed software image. To counter this threat, Cubedate uses digital signatures on a hash of the image binary and the metadata to ensure integrity of both the firmware and its metadata.

\item {\bf Unauthorized Software Update Attacks –} An unauthorized party may attempt to update the IoT device with modified image. Using digital signatures and public key cryptography, Cubedate ensures, that only the authorized maintainer (holding the authorized private key) will be able to update de device.
\end{itemize}

Going beyond simple authenticity and integrity guarantees for software updates delivered over the network, using Cubedate also mitigates other attacks including:
\begin{itemize}
\item {\bf Software Update Replay Attacks –} An attacker may try to replay a valid, but old (known-to-be-flawed) software. This threat is mitigated by using a sequence number. Cubedate uses a sequence number, which is increased with every new software update.

\item {\bf Software Update Mismatch Attacks –} An attacker may try replaying a software update that is authentic, but for an incompatible device. Cubedate includes device-specific conditions, which can be verified before installing a software binary, thereby preventing the device from attempting to use an incompatible software image.
\end{itemize}

\iffalse

\paragraph*{communication protocols}
\paragraph*{software/firmware updates}
\paragraph*{software/configuration updates requirements}

\subsection{Architecture}
\subsubsection{OS}: RIOT
\subsubsection{Network Stack}: COAP/LibCSP/CAN
\subsubsection{Mission Workflow}: mission files
\subsubsection{DevOps Workflow}: SUIT + containers
\paragraph*{SUIT}
\paragraph*{Containerization}
\fi

\section{Implementation}
\label{sec:implementation}

\subsection{Payload}
\label{sec:paylod}
\subsubsection{Hardware}
\paragraph*{MCU} 
\paragraph*{LoRa Radio} 
\paragraph*{LoRa Gateway} 

\subsubsection{Firmware}
\paragraph*{libCSP} 
\paragraph*{CoAP}  % or SFP + RDP...
\paragraph*{CAN} 
\paragraph*{LoRaWAN} 
\paragraph*{Application Code}
% - Heartbeat
% - Mission task
% - Watchdog
% - Two Line Element
\subsubsection{Code Structure}

\subsection{OBC Simulator}

\subsection{Continuos Integration - Leveraging OpenSource}
\paragraph*{RIOT}
% - pointers to simulator-payload integations
% - LibCSP in RIOT
% - SUIT workflow examples in RIOT
% - container examples in RIOT
% - LoRaWAN examples in RIOT
\paragraph*{ThingSat}
% - OBC & Payload integration tests
% - gitlab CI build tests and others


\section{Cubedate Evaluation and Discussion}
\label{sec:evaluation}

The analysis of the implementation was two-fold, first we evaluate how the SUIT
based architecture fitted the CubeSat and CubeSat hosted payload use case.
Second, we discuss possible modifications to the workflow, architecture and
protocols.

\subsection{Evaluation}

In the following, code measurements where generated compiling with ARM GCC 10.2.1,
optimized for code size. The crypto library used is D.Beer C25519 library, this
is the library with the lowest footprint for ED25519 as shown in~\cite{zandberg2019secure}.

\subsubsection{Memory Footprint}

To evaluate the RAM and flash footprint of the CubeDate architecture we evaluate them
in two configurations, furthermore we categorize the different components to
distinguish the SUIT endured overhead

\begin{itemize}
    \item \textbf{ThingSat} refers to the ThingSat payload application with no
    software updates support
    \item \textbf{CubeDate} implementation of the CubeDate architecture on the
    ThingSat payload
    \item CAN (Controller Area Network) stack as well as low level interface
    \item Crypto includes all cryptographic algorithms such as digest algorithm, digital
    signature, ECC and bignum, as well as pseudo-random numbers generator
    \item CoAP protocol library (CoAP endpoint stack excluded)
    \item CSP (Cube Sat Protocol) network stack
    \item LoRa GW includes the sx1302 driver as well as the gateway code
    \item SUIT englobe all components enabling retrieval and installation of suit data
    (fw or other), this include e.g. the CoAP endpoint stack.
    \item Firmware: application specific code related to the CubeSat Payload excluding
    the LoRa gateway
\end{itemize}

The flash memory footprints (total and broken down per component) are shown in
\autoref{tab:footprint}. It can be seen that the overhead of Cubedate for ThingSat
is of \~4Kbytes of RAM and \~19Kbytes of flash.

In \autoref{tab:manifest-overhead} network transfer overhead is measured against the average mission file size (700Bytes) and the average firmware image size (200 KBytes).

\begin{table}[ht]
\begin{adjustbox}{width=0.8\columnwidth,center}
    \centering
    \includestandalone[width=1\columnwidth]{Figures/texfigs/memory_footprint_table}
\end{adjustbox}
\caption{Cubedate implementation: memory footprint in Bytes.}
\label{tab:footprint}
\end{table}

\subsubsection{Standarized Network Stack Usage}

CSP was designed to give CubeSat sub-systems developers the same features as an
TCP/IP stack without the overhead of the IP header, allowing to run on constrained
systems with under 4Kbytes of RAM. Unless running on this kind of very constrained
devices using CSP can be contested. A minimal CoAP server example running on
LibCSP or RIOTs default IPv6/UDP network stack (GNRC) yields similar numbers of RAM/Flash
usage: 10KB\/30KB (CSP) vs 8KB/31KB (GNRC). Optimal compression of an Ipv6 header
can shrink the size from 48bytes to 4Bytes, comparable to CSP2.0 3Bytes header.
Furthermore  6loCAN\cite{wachter20206locan01}, SCHC\cite{rfc8724}, allow to
optimize IPv6 for CAN-BUS or LoRa network. What is gained in simplicity is eventually
lost by using a not standard network stack, i.e. using standard application
layer protocols.

\begin{table}[ht]

\begin{adjustbox}{width=0.5\columnwidth,center}
    \centering
    \includestandalone[width=1\columnwidth]{Figures/texfigs/manifest_overhead_table_simplified}
\end{adjustbox}
\caption{Cubedate implementation: SUIT metadata overhead.}
\label{tab:manifest-overhead}
\end{table}

\subsubsection{SUIT Authentication}

SUIT relies on COSE for authentication, with digital signatures being the most
used method. A viable alternative for CubeSat maybe to instead use a tagged Message
Authentication Code (MAC). This will reduce the size of \textit{Authentication} block
\ref*{tab:manifest-overhead} from 64Bytes to 16Bytes or 8Bytes if using HAMC-256/128
or HMAC-256/64. This would also save Flash since as seen in~\cite{zandberg2019secure}
the ed25519 library accounts for roughly 75\% of the crypto flash budget, while HMAC-256
re-uses sha256 code, and adds an overhead of ~1Kbyte. The requirement for a shared secret
might be an acceptable compromise since attackers gaining physical access to the
in-orbit device is unlikely.

\subsubsection{SUIT Manifest Overhead}

In space communication transferring data is costly, reducing the Manifest
overhead~\ref{tab:manifest-overhead} for mission files transfers is highly
desirable. This can be done by using \texttt{COSE\_MAC0\_tagged} instead of
\texttt{COSE\_SGN0\_tagged} (-56Bytes) and integrating the payload into the
manifest (-64Bytes from the URI), reducing the manifest size to 162Bytes,
effectively reducing the overhead to \~23\%.



\section{Next Steps \& Future Work}
\label{sec:futurework}

\paragraph*{RIOT/SUIT differential firmware updates}
\paragraph*{Business logic containerization}

\section{Conclusion}
\label{sec:conclusion}


%\section*{Acknowledgment}

\bibliographystyle{IEEEtran}
\bibliography{biblio}

\end{document}
