\documentclass[conference]{IEEEtran}
% Some Computer Society conferences also require the compsoc mode option,
% but others use the standard conference format.
%

% *** CITATION PACKAGES ***
%
\usepackage{cite}
% cite.sty was written by Donald Arseneau


% *** GRAPHICS RELATED PACKAGES ***
%
\ifCLASSINFOpdf
  \usepackage[pdftex]{graphicx}
  % declare the path(s) where your graphic files are
  \graphicspath{{../pics/}}
  % and their extensions so you won't have to specify these with
  % every instance of \includegraphics
  %\DeclareGraphicsExtensions{.pdf,.jpeg,.png}
\else
  % or other class option (dvipsone, dvipdf, if not using dvips). graphicx
  % will default to the driver specified in the system graphics.cfg if no
  % driver is specified.
  \usepackage[dvips]{graphicx}
  % declare the path(s) where your graphic files are
  \graphicspath{{./pics}}
  % and their extensions so you won't have to specify these with
  % every instance of \includegraphics
  \DeclareGraphicsExtensions{.eps}
\fi

\usepackage{comment}
\usepackage{mcite}
\usepackage{cancel}
\usepackage{blkarray}
\usepackage[export]{adjustbox}
\usepackage{tikz}
\usepackage[mode=buildnew]{standalone}% requires -shell-escape
\usepackage{pgfplots, pgfplotstable}
\pgfplotsset{compat=1.17}
\usepackage{lipsum}
\usepackage{colortbl}

\newcommand{\francisco}[1]  {\textbf{\textcolor{blue}{[Francisco] #1}}}
\newcommand{\nath}[1]{\textcolor{green}{#1} }

% *** MATH PACKAGES ***
%
\usepackage{amsmath}

% *** SUBFIGURE PACKAGES ***
\ifCLASSOPTIONcompsoc
  \usepackage[caption=false,font=normalsize,labelfont=sf,textfont=sf]{subfig}
\else
  \usepackage[caption=false,font=footnotesize]{subfig}
\fi

\begin{document}
\bstctlcite{IEEEexample:BSTcontrol}

\title{ThingSat, Open-Source CubeSat Payload Deployment}

% author names and affiliations
% use a multiple column layout for up to three different
% affiliations
\author{
    %\IEEEauthorblockN{J. Kirk, H. Simpson and M. Scott}
    \IEEEauthorblockN{Authors}
    \IEEEauthorblockA{Something, France.
        Email: first.lastname@something.fr}}

% make the title area
\maketitle

% As a general rule, do not put math, special symbols or citations
% in the abstract
\begin{abstract}

\end{abstract}

\IEEEpeerreviewmaketitle

\section{Introduction}
\label{sec:introduction}

\paragraph*{Satellites, CubeSat}
\paragraph*{IoT, Internet of Space Things}: Iridium, LoRa, 5G
\paragraph*{Cube-Sat, Open-Source Initiatives}
\paragraph*{RIOT, Open-Source}
\paragraph*{Paper contributions}
\paragraph*{Paper organization}

\section{Background \& Related Work}
\label{sec:background-related-work}

\subsection{CubeSat}
\label{sec:cubesat}

\subsection{Cube Sat Protocol (CSP)}
\label{sec:csp}

\paragraph*{Architecture}
% - electronics
% - connectivity
% - payloads, obc, etc..

\subsection{Open Source}\cite{shalashov2021OpenSourceCubeSatReview}, \cite{Holliday2019PyCubedAO}
\paragraph*{Why Open-Source Satellite Initiatives}
\paragraph*{pyCube, etc.}
% - overview of alternatives
% - hardware software overview

\subsection{LoRa} \cite{saeed2020CubeSatReview}
\label{sec:lora-cubesat}
\paragraph*{CubeSat LoRa missions}
\paragraph*{Research topics, challenges of LoRa in space}

\subsection{RIOT} \cite{baccelli2018riot}
\paragraph{Overview of the Project}
\paragraph{Overview of the architecture}

\section{ThingSat}
\label{sec:case-study}

\subsection{Overview}

\paragraph*{Project timeline and participants}
% - costs
\paragraph*{Mission Goal}
\paragraph*{Broad overview of Segments}

\subsection{Architecture}
\paragraph*{Ground Segment Description}
\paragraph*{Space Segment Description}
\paragraph*{Control Segment} % Francisco: not sure if its the right name
\paragraph*{OBC \& Payload Communication Bus}
% - OBC -> SatRev controlled
% - Payload -> Full Control

\subsection{Telecommunications}
\paragraph*{link-budget Mission Control}
\paragraph*{link-budget Mission}
\paragraph*{available phy}

\section{Low Power Orbital Communication Architecture}
\label{sec:low-power-orbital-communicaiton-arch}

\subsection{Requirements}
\paragraph*{communication protocols}
\paragraph*{software/firmware updates}
\paragraph*{software/configuration updates requirements}

\subsection{Architecture}
\subsubsection{OS}: RIOT
\subsubsection{Network Stack}: COAP/LibCSP/CAN
\subsubsection{Mission Workflow}: mission files
\subsubsection{DevOps Workflow}: SUIT + containers
\paragraph*{SUIT}
\paragraph*{Containerization}

\section{Implementation}
\label{sec:implementation}

\subsection{Payload}
\label{sec:paylod}
\subsubsection{Hardware}
\paragraph*{MCU}
\paragraph*{LoRa Radio}
\paragraph*{LoRa Gateway}

\subsubsection{Firmware}
\paragraph*{libCSP}
\paragraph*{CoAP}  % or SFP + RDP...
\paragraph*{CAN}
\paragraph*{LoRaWAN}
\paragraph*{Application Code}
% - Heartbeat
% - Mission task
% - Watchdog
% - Two Line Element
\subsubsection{Code Structure}

\subsection{OBC Simulator}

\subsection{Continuos Integration - Leveraging OpenSource}
\paragraph*{RIOT}
% - pointers to simulator-payload integations
% - LibCSP in RIOT
% - SUIT workflow examples in RIOT
% - container examples in RIOT
% - LoRaWAN examples in RIOT
\paragraph*{ThingSat}
% - OBC & Payload integration tests
% - gitlab CI build tests and others

\section{Evaluation}
\label{sec:evaluation}

\subsection{Memory Footprint}
\subsection{Energy Footprint}
\subsection{Link-budget Footprint}
\subsection{Updates Latency}

\section{Lessons Learned}

\section{Next Steps \& Future Work}
\label{sec:futurework}

\section{Conclusion}

\section*{Acknowledgment}

\bibliographystyle{IEEEtran}
\bibliography{IEEEabrv,./biblio}

\end{document}
