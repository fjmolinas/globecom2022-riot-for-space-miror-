\section{Low Power in-Orbit Continuos Deployment Architecture}
\label{sec:low-power-orbital-communicaiton-arch}

The process we use for securing ThingSat software udpates is decomposed in six phases shown \autoref{fig:phases}. During a preliminary phase (\textit{Phase 0}) the authorized maintainer for the CubeSat hosted payload
produces and flashes the payload with commissioning material:
the bootloader, the initial image, and authorized crypto material (a public key, and a crypographically strong hash function).
Once the hosted payload is commissioned it can be sent to the CubeSat operator of installation in the CubeSat.
Once the CubeSat in orbit, the hosted payload maintainer can trigger iterations through cycles of phases 1-5, whereby
the authorized maintainer can build a new software update (\textit{Phase 1}), hash
and sign the update (\textit{Phase 2}) and initiate a network transfer (PUT) to hosted payload via the ground station and the OBC. The hosted payload can
then fetch the update from the OBC (\textit{Phase 3}), proceed to verify the signature and the hash (\textit{Phase 4}),
and upon successful verification, the new software is installed and
booted (\textit{Phase 5}), otherwise the update is dropped.

\begin{figure}[t]
    \centering
    \includegraphics[width=0.5\textwidth]{Figures/CubeSat-Payload-update.png}
    \caption{CubeSat hosted payload secure software update process.}
    \label{fig:phases}
\end{figure}


\subsection{Requirements}
\paragraph*{communication protocols}
\paragraph*{software/firmware updates}
\paragraph*{software/configuration updates requirements}

\subsection{Architecture}
\subsubsection{OS}: RIOT
\subsubsection{Network Stack}: COAP/LibCSP/CAN
\subsubsection{Mission Workflow}: mission files
\subsubsection{DevOps Workflow}: SUIT + containers
\paragraph*{SUIT}
\paragraph*{Containerization}

