\section{Introduction}
\label{sec:introduction}

Driven in parts by the increasing involvement from public research and the academic
community~\cite{cubesat101}, more and more open source software and hardware is available,
maturing and used on CubeSats.

The availability of such a basis tends to lower the bar of entry, with the view to
designing low-cost CubeSats which can more reliably achieve a successful launch --
until recently, the failure rate was still around 60\% for first-time satellite
builders~\cite{Holliday2019PyCubed}.

Another trend aiming to lower the bar of entry is to buy tiny "rack-space" in
orbit~\cite{satrevolution2020}. With this model, a stakeholder (a user) can place
a small ($<$1U) payload slot hosted inside a CubeSat provided by an operator
(a different stakeholder). The user only needs to design and operate this payload,
instead of designing and operating the whole CubeSat, which drastically reduces users'
costs and lead time. Typically, such a payload boils down to a printed circuit board
(PCB) with sensors, a low-power CPU (sometimes with separate wireless communication
capabilities) and a connector to the local bus communication to the main on-board
flight computer (OBC) on the CubeSat. As such, a hosted payload on a CubeSat
resembles small embedded devices usually found in the Internet of Things (IoT).

Nevertheless, even with an entry bar lowered by combining a hosted payload approach
with leveraging open source, software embarked on launched CubeSats tends to be
minimalist (think: space rush, time pressure to meet hard launch deadlines...
and buggy, as most software is. In effect, hosted CubeSat payload software must
typically be updated over-the-air (OTA) regularly, after it is deployed in orbit
and in operation. Beyond enabling (OTA) software updates on a hosted CubSat payload,
cybersecurity is an crucial aspect which emerges. Indeed, on the one hand software
updates can  be used to fix vulnerabilities but on the other hand tampering with
software updates can be used as cyberattack vector~\cite{ccleaner}.

State-of-the-art has so far focused on single-stakeholder CubeSats use cases, where the software embarked on the CubeSat is managed by a signle entity. In
this paper, we instead focus on the challenge of securing software updates on low-cost
multi-tenant CubeSats, whereby the OBC and the payload are operated by different
stakeholders which do not necessarily trust each other. In particular, we focus on payloads
based on low-power microcontrollers, which are essential on low-cost CubeSats, where
low power consumption is key.

The contributions of out work are two-fold. We present an architecture for secure
and standards based end-to-end software update on low-power microcontroller based
CubeSat payloads. This solution is motivated and designed against ThingSat, a
deployed CubeSat case-study showing the heterogeneity of update targets and
involved network transports. We then implement this architecture for ThingSat,
and released these developments in benefit of the RIOT OpenSource community
as well the CubeSat community by providing a base for a collaborative continuous
integration platform allowing to test and verify the update-ability and functionality
of low-power embedded CubeSat payloads.


\iffalse

\textit{Paper contributions:}
\begin{itemize}
    \item We describe ThingSat as case-study for hosted CubeSat payload and we analyse its requirements and contraints in terms of in-orbit software updates;
%    \begin{itemize}
%        \item shows heterogeneity of update targets (firmware, software, configs)
%        \item using diverse network transports and through different intermediaries
%    \end{itemize}
    \item We define XXXX (find a name here?) an architecture for secure and standards based end-to-end software updates for low-power MCU CubeSat payloads;
    \item We provide and evaluate an open source implementation of our XXXX architecture;
    \item ??? We provide a basis for a collaborative continuous integration platform allowing to test and verify the continuous deployment chain.???
\end{itemize}

\textit{Paper Organization}
TBD.




\textit{IoT, Internet of Space Things}: Iridium, LoRa, 5G

\textit{Satellites generics}

Education origin of satellite projects, with goal of lowering entry barrier by
lowering design complexity. Cost have been also reduced with the increased
amount of CubeSat deployment, numbering handful before 2013, +150 on average since,
and +500 projected. The success rate has also increased over the years but is still
around 75\%\cite{villela2019towards1000} (less for education??) and the failure
rate still being around 60\% for first time satellite builders\cite{Holliday2019PyCubed}.

The nature of student driven CubeSat projects: changing teams, limited time,
priorities conflicts, parallel hardware and software development mean that delivery
and launch deadlines are not always met with the adequate level of satellite software
testing.

Building on the experience of other projects can reduce development times, allowing
for more time to be allocated for software testing in an operational scenario. OpenSource
allows such approach with projects like PyCubed\cite{Holliday2019PyCubed} and others
\cite{shalashov2021OpenSourceCubeSatReview}, mostly focusing on Linux distributions.
\textit{RIOT, Open-Source}

Bugs will still be present, and even in some cases new or missing functionality
might need to be implemented and deployed after launch. In-orbit software updates
is common and standard in Large Scale Satellites but is much less documented
for nanosatellites,\cite{sunter2016updatesnano} \cite{maison2021otaeducubesat}.
But in most cases these all consist on firmware updates, cases where individual
software modules or business logic are updated have not been studied. They seldom
talk about security and authenticity.

The ThingSat project was such scenario, the inherent challenges of any Education
CubeSat project where intensified by the COVID pandemic. Experiment and mission
firmware will simple not be able to be delivered on time (X amount of time between
CubeSat payload delivery and lunch).

Payload last point in Ground-Flight chain, with little to no control over what
happens in the middle. With tight regulation over space communication (noise),
risk of hacking, etc. any firmware update or software update delivered to the
payload must be checked for corruption but also tampering. Different subsystems
requiring updates by different stakeholders.

Ground-Flight communication increases the challenges here because of the limited
bandwidth, high latency, etc.
\paragraph*{Challenges of firmware updates in space}: link budget, memory corruption
due to space radiation, high latency, etc.

\fi
