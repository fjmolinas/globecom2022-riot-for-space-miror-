\section{Introduction}
\label{sec:introduction}

Driven in part by the increasing involvement from public research and the academic
community~\cite{cubesat101}, more and more open source software and hardware is available,
maturing and used on CubeSats.

Their availability has lowered the bar of entry, with the aim of designing low-cost
CubeSats which can more reliably achieve a successful launch -- until recently, the
failure rate was still around 60\% for first-time satellite builders~\cite{Holliday2019PyCubed}.

With the same objective, another trend has been to buy tiny "rack-space" in
orbit~\cite{satrevolution2020}. With this model, a stakeholder (a user) can place
a small ($<$1U) payload slot hosted inside a CubeSat provided by an operator
(a different stakeholder). The user only needs to design and operate this payload,
instead of designing and operating the whole CubeSat, drastically reducing users'
costs and lead time. Typically, such a payload boils down to a printed circuit board
(PCB) with sensors, a low-power CPU (sometimes with separate wireless communication
capabilities) and a bus communication interface to the CubeSat main on-board flight
computer (OBC). As such, a hosted payload on a CubeSat resembles small embedded
devices usually found in the Internet of Things (IoT).

Nevertheless, even if the entry bar is lowered through the combination of a hosted payload
and leveraging open source, software embarked on launched CubeSats tend to be
minimalist (think: space rush, time pressure to meet hard launch deadlines...)
and buggy, as most software is. In effect, hosted CubeSat payload software must
typically be updated over-the-air (OTA) regularly, after it is deployed in orbit
and in operation. Beyond enabling OTA software updates on a hosted CubSat payload,
cybersecurity is a crucial emerging aspect. Indeed, on the one hand, software
updates can be used to fix vulnerabilities but, on the other hand, tampering with
software updates can be used as a cyberattack vector~\cite{ccleaner}.

State-of-the-art has so far focused on single-stakeholder CubeSats use cases,
where embarked software is managed by a single entity. In this paper, we instead
focus on the challenge of securing software updates on low-cost multi-tenant CubeSats
whereby the OBC and the payload are operated by different stakeholders which do not
necessarily trust each other. In particular, we focus on payloads based on low-power
microcontrollers, which are essential on low-cost CubeSats, where
low power consumption is a key factor.
