\section{Introduction}
\label{sec:introduction}

\textit{IoT, Internet of Space Things}: Iridium, LoRa, 5G

\textit{Satellites generics}

Education origin of satellite projects, with goal of lowering entry barrier by
lowering design complexity. Cost have been also reduced with the increased
amount of CubeSat deployment, numbering handful before 2013, +150 on average since,
and +500 projected. The success rate has also increased over the years but is still
around 75\%\cite{villela2019towards1000} (less for education??) and the failure
rate still being around 60\% for first time satellite builders\cite{Holliday2019PyCubed}.

The nature of student driven CubeSat projects: changing teams, limited time,
priorities conflicts, parallel hardware and software development mean that delivery
and launch deadlines are not always met with the adequate level of satellite software
testing.

Building on the experience of other projects can reduce development times, allowing
for more time to be allocated for software testing in an operational scenario. OpenSource
allows such approach with projects like PyCubed\cite{Holliday2019PyCubed} and others
\cite{shalashov2021OpenSourceCubeSatReview}, mostly focusing on Linux distributions.
\textit{RIOT, Open-Source}

Bugs will still be present, and even in some cases new or missing functionality
might need to be implemented and deployed after launch. In-orbit software updates
is common and standard in Large Scale Satellites but is much less documented
for nanosatellites,\cite{sunter2016updatesnano} \cite{maison2021otaeducubesat}.
But in most cases these all consist on firmware updates, cases where individual
software modules or business logic are updated have not been studied. They seldom
talk about security and authenticity.

The ThingSat project was such scenario, the inherent challenges of any Education
CubeSat project where intensified by the COVID pandemic. Experiment and mission
firmware will simple not be able to be delivered on time (X amount of time between
CubeSat payload delivery and lunch).

Payload last point in Ground-Flight chain, with little to no control over what
happens in the middle. With tight regulation over space communication (noise),
risk of hacking, etc. any firmware update or software update delivered to the
payload must be checked for corruption but also tampering. Different subsystems
requiring updates by different stakeholders.

Ground-Flight communication increases the challenges here because of the limited
bandwidth, high latency, etc.
\paragraph*{Challenges of firmware updates in space}: link budget, memory corruption
due to space radiation, high latency, etc.

\textit{Paper contributions, Paper organization}
